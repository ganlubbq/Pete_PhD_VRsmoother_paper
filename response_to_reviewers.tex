\documentclass{article}

\usepackage{amsmath}
\usepackage{IEEEtrantools}
\usepackage{color}

\newenvironment{meta}[0]{\color{red} \em}{}

\begin{document}

{\meta Many thanks for the comments, folks. I've made changes to address them all. Responses are inline with the individual points below. }

\section*{Reviewer Comments}

\subsection*{Reviewer: 1}

Recommendation: AQ - Publish In Minor, Required Changes

\subsubsection*{Comments}
This is an interesting, well written paper  that suggests new sampling strategies for smoothing  in variable rate models. The proposed method  rejuvenates the variable rate filtering particles that usually have poor smoothing performance due to the loss of particle diversity particularly when state discontinuities occur.

The paper relays on the key idea, using the previously generated filtering particles as a proposal  to the target conditional distributions, which is inherited from the previous work of the author [Godsill et al. 2004] . The proposed algorithms apply a backward pass by sampling the  target conditional distribution $P( \theta^{-}_{n} |  . )$  for $n=N�1$.   Two methodologies are proposed for sampling the target conditional distribution of the  variable rate structured "conditionally linear Gaussian models"  and "conditionally deterministic models". The former samples from a discrete set of re-weighted filtering particles according to well known Rao-Blackwellization approach and the other uses an  MCMC sampling scheme where the proposals are constructed by a  modified  filtering particle set.

Both algorithms are sophisticated techniques for rejuvenating the particles and can be  interpreted as a way of reconstructing the particle history by using the filtering particles sampled at each filtering step. Reference on the convergence property of the algorithms and a  discussion to understand the practical aspects of the the algorithm (Ex. whether is it  possible to increase the online filtering performance by using retrospective smoothing steps, the ability of the algorithm to  find the untracked modes of the posterior - this can also be suported with simulation results )  would be useful  for the reader in order to understand the limits of the algorithms.

{\meta I've added a brief note on convergence, which is inherited from the basic results for the forward-filtering-backward-sampling algorithm. I've also added an explicit example in the simulations of the smoother representing multi-modality where some particle contain a jump and others don't. Fixed interval smoothing won't help online filtering. Fixed-lag smoothing would, and this is covered elsewhere, especially in [Whitely et al. 2011]. }

In the simulations section, smoothing performance of the algorithms are compared to filtering and  the filtering- smoothing results.  In both tests improvement is achieved in means of RMSE. I see that in the  simulations filter and filter- smoother is able to represent the solution to the data. However it would be much more interesting and convincing to see an illustration of a scenario  where the filter is not able to represent the solution due to discarded particles in the resampling steps  (Ex.  poor estimate for the changepoint location or  parameters,  misses a jump or unable to track the maneuver due to multi-modality)  but  the smoother is able track the solution.

{\meta Because the smoother uses filter particles as a proposal, if the filter fails catastrophically (e.g. track loss) then the smoother will also fail. I have added an example where the multi-modality of the particle representation is lost due to resampling in the filter-smoother, but not the filter. }


Minor comments:

Page: 37 line 36 :  " $\theta$ " should be  "  $\theta'$ "
Fig: 4 step 11 " $\theta*$ " should be  " $\theta'*$  "

{\meta Well spotted. Now correct.}



\subsection*{Reviewer: 2}

Recommendation: A - Publish Unaltered

\subsubsection*{Comments}
(There are no comments. Please check to see if comments were included as a file attachment with this e-mail or as an attachment in your Author Center.)

{\meta No file attached, so I assume there are genuinely no comments. }



\subsection*{Reviewer: 3}

Recommendation: RQ - Review Again After Major Changes

\subsubsection*{Comments}
The authors propose a smoothing PF for variable rate models. The proposed method is essentially an adaptation of existing methods for fixed-rate models to the variable rate model. I feel there is enough substance in the adaptation process to merit publication provided some revisions are performed. My main concerns are with the performance analysis:

- It is never clarified but I'm assuming that the smoothing results are for fixed-interval smoothing, i.e., estimates at time $t_n, n=1,...,N$ use measurements from times $t_1,...,t_N$. This should be clearly stated.

{\meta I feel this is clear from the introduction and the maths, but it's now explicitly stated as well. }

- Only 10 realisations are used! It's difficult to draw conclusions with such a small sample size.

{\meta Ok. Now 100.}

- In Figs 7 and 10 the effective sample size would be a better measure of performance than the number of unique particles.

- The principal improvement offered by the more sophisticated particle smoothers compared to the filter-smoother is increased particle diversity. This should improve estimation performance but the improvements in consistency may be even greater. This is alluded to in the paper and partially covered by Figs 6 and 9 which show plots of sample trajectories. However, a more rigorous consistency analysis should be performed. For example, a simple approach would be to construct a Gaussian approximation to the posterior, establish a confidence ellipsoid and then count the number of times the true value falls within this ellipsoid.

{\meta These two comments concern the same question: How should particle approximations of a distribution be compared? There is, of course, much more to a distributional estimate than how close its mean is to the true value, so we do need something more than just the RMSE, as you suggest. One useful and established metric for measuring efficiency is the normalised estimation error squared (NEES), but we can't use this directly, because when the particle estimate is degenerate the NEES is infinite. Therefore, I've added a transformed version of the NEES as an efficiency measure. I think that together with the RMSE values, this provides sufficient evidence of the improvement provided by the smoother. }

- Comparing the proposed method and the filter-smoother for equal sample sizes doesn't really make sense given that the former has $O(N^2)$ expense while the latter has $O(N)$ expense. The filter-smoother should be implemented with a much larger sample size such that both algorithms have approximately equal expense. This approach, which is taken in (Fearnhead et al, 2010), seems much fairer.

{\meta I think this is a bit misleading really. Although it might be a fairer test for a given problem, the result can be changed by modifying the scenario. For example, the relative performance of the smoother will be improved by having a longer data set. All I want to demonstrate in the simulations section is that the smoother operates as expected --- producing rejuvenated particle sets. The better algorithm for any application will depend on the model, the length of the data, and the desired output (e.g. is a distributional estimate needed, or is a single sample enough?).

In contrast to us, [Fearnhead et al. 2010] are comparing a variety of smoothing algorithms all of which do the same thing, produce a diverse set of marginal smoothing particles. Simple changes like altering the data length are less likely to affect their results, so their methods are appropriate to their purpose, but less so here.

However, Simon agrees with you. So i have actually changed the number of particles in the comparisons based on this equal-computation principle. }

- Results should be shown for a range of sample sizes so we can get an idea of the required sample size for (close to) optimal performance.

{\meta Yes. I've added graphs showing the effect of varying the number of smoothing trajectories for the tracking case. The finance results are similar and so omitted for space. }


Some minor comments:
- A Rao-Blackwellised VRPF was also developed in (Morelande and Gordon, 2009). Their formulation was more general than that used here since it permits the model to be partially linear-Gaussian.

{\meta Yes. This should be and now is referenced. Actually, their formulation is not more general, just different. The Rao-Blackwellisation in our paper is of the state variables when the dynamics are linear-Gaussian, while [Morelande and Gordon, 2009] Rao-Blackwellise the linear parts of the motion parameters, but in a conditionally deterministic model. We could extend our work by using their Rao-Blackwellisation scheme on the piecewise-deterministic smoother. }

- Recently, O(N) schemes for particle smoothing which improve upon the filter-smoother have been proposed by (Fearnhead et al, 2010) and (Douc et al, 2011). Perhaps these could be used for VR filtering. They should at least be mentioned.

{\meta I've added a reference to [Douc et al., 2011]. This is similar to our Metropolis-Hastings-based method which is already referenced (but unhelpfully so as it is still awaiting publication). ([Bunch and Godsill, 2012]). The Fearnhead method is for marginal (as opposed to joint) smoothing distributions only, so it doesn't really help. I've mentioned this in section IV. }

- My understanding is that in (33) $\tau_{K_{n-1}+1}$ should be drawn from a truncated distribution which forces it to be greater than $t_{n-1}$. According to the notation of (4), shouldn't the denominator then be $S(\tau_{K_{n-1}},...)$ rather than $S(\tau_{K_{n-1}+1},...)$?

{\meta Yes. Well spotted! Corrected. }

- On page 13, just before (51), there is a $\tilde{\theta}$ which should be a $\theta$.

{\meta Fixed. }

- In Fig. 9(a), are these results for the filter-smoother?

{\meta Yes. Clarified.}

- I'm not sure all the appendices are necessary since the results can be found in other, easily obtained sources. At least

{\meta I agree. I've cut the first two, giving the derivation and initialisation of the two-filter smoother. The discretisation of the finance model remains, as I can't find a complete and equivalent version in a published work. }



\end{document} 